\documentclass[letterpaper,sansserif,tightsqueeze]{module}

\usepackage{parskip}                                                            % Add spacing between paras instead of indents

\title{B1: In Search of the Unknown}

% Compress title spacing compared to default

\addtolength{\topmargin}{-0.3cm}
\addtolength{\textheight}{0.7cm}

% Initialise counters

\setcounter{page}{20}

\begin{document}

\onecolumn

\begin{center}
Page intentionally left blank.
\end{center}

\twocolumn

\section*{KEY TO THE LOWER LEVEL}

The lower level of the complex is rough and unfinished. The
walls are irregular and coarse, not at all like the more finished
walls of the level above (except for the two rooms on
this level which are more like those in the upper portion and
in a state of relative completion). The corridors are roughly
10' wide, and they are irregular and rough, making mapping
difficult. The floors are uneven, and in some cases rock
chips and debris cover the pathways between rooms and
chambers. The doors are as in the upper level, but the secret
doors are either rock or disguised by rock so as to appear
unnoticeable.

\section{WANDERING MONSTERS}

Check every second turn; 1 in 6 (roll a 6-sided die). If a monster
is indicated, roll a six-sided die again and compare to
the list below to determine what type of monster appears.
Then check for surprise. The abbreviations which follow are
the same as used and explained in the section entitled MONSTER
LIST.
\begin{enumerate}
\item\stats{troglodyte}{1--4}{9,8,5,4}
\item\stats{crab_spider}{1}{12}
\item\stats{kobold}{2--7}{4,4,3,3,2,2,1}
\item\stats{orc}{1--8}{6,5,5,4,4,3,3,2}
\item\stats{zombie}{1--2}{8,7}
\item\stats{goblin}{2--7}{5,5,4,4,3,2,1}
\end{enumerate}

\section{ENCOUNTER AREAS}

\setcounter{subsection}{37}

\subsection{ACCESS ROOM}
This room is filled with piles of rock and
rubble, as well as mining equipment: rock carts, mining
jacks, timbers, pickaxes, etc. It is apparent that there has
been no mining activity for quite some time.

Monster:

Treasure \& Location:

\subsection{MUSEUM}
This room is an unfinished museum, a special
monument to the achievements of the stronghold's most illustrious
inhabitants.

The west wall is a sectioned fresco showing various events
and deeds from the life of Rogahn, and the several views
pictured are: a young boy raising a sword, a young man
slaying a wild boar, a warrior carrying off a dead barbarian,
and a hero in the midst of a large battle hacking barbarian
foes to pieces.

The east wall is a similar sectioned fresco showing cameos
from the life of Zelligar: a boy gazing upward at a starry night
sky, a young man diligently studying a great tome, an earnest
magician changing water to wine before a delighted
audience, and a powerful wizard casting a type of death
fog over an enemy army from a hilltop.

The north wall section is unfinished, but several sections of
frescoes show the two great men together: shaking hands for
the first time in younger days, winning a great battle against
barbarians in a hill pass, gazing upward together from the
wilderness to a craggy rock outcropping (recognizable to
the adventurers as the place where the stronghold was built),
with a fourth space blank. Next to the frescoes are other mementoes
from the past: a parchment letter of thanks for help
in the war against the barbarians from a prominent landowner,
a barbarian curved sword, and a skeleton of the barbarian
chief (so identified by a wall plaque in the common
language). There is more blank space on the wall, apparently
for further additions to the room's collection of items.

The frescoes are painted and they cannot be removed.
None of the mementoes is of any particular worth or value.

Monster:

Treasure \& Location:

\section{40--56. CAVERNS OF QUASQUETON}

\setcounter{subsection}{39}

The bulk of the lower
level of the complex is a series of unfinished caves and caverns,
which are mostly devoid of special detail—all being
characterized by irregular walls of rough rock. Uneven floors
strewn with bits of rock and rubble, and joined by winding
corridors. The majority of the rooms are empty of furnishings.

\subsection{SECRET CAVERN}

Monster:

Treasure \& Location:

\subsection{CAVERN}

Monster:

Treasure \& Location:

\subsection{WEBBED CAVE}

The entrance to this room is covered
with silky but sticky webs, which must be cut or burned
through to gain access to it. See web spell for details in D\&D
Basic booklet.

Monster:

Treasure \& Location:

\subsection{CAVERN}

Monster:

Treasure \& Location:

\subsection{CAVERN}

Monster:

Treasure \& Location:

\subsection{CAVERN OF THE MYSTICAL STONE}
This ante-chamber is
the resting place for a large, glowing chunk of rock which
appears to be mica. The stone radiates magic strongly.

The stone rests permanently in its place and is not removable.
Although chips can easily be broken off the rock by
hand, only one chip at a time may be broken away; until
anything is done with it, the rest of the rock will remain impervious
to breaking.

Once a chip is removed, its glow will begin to fade, and after
three rounds (thirty seconds) it will be a normal piece of mica
with no magical properties (as will be the case if it is removed
from this room). The chip's magical properties are manifested
only if it is consumed (or placed in the mouth) by any
character before three rounds have passed after breaking
off from the chunk. The magical effects are highly variable
and each individual can only be once affected—even if a
future return to the rock is made at a later time. If any character
places a chip within his or her mouth, a 20-sided die is
rolled to determine the effect according to the following table:

\begin{enumerate}
\item Immediately teleports the character and his gear to the
webbed cave (room 42)
\item Immediately blinds the character for 1--6 hours of game
time (no combat, must be led by other adventurers)
\item Raises strength rating permanently by 1 point
\item Raises charisma rating permanently by 1 point
\item Raises wisdom rating permanently by 1 point
\item Raises intelligence rating permanently by 1 point
\item Raises dexterity rating permanently by 1 point
\item Lowers strength rating permanently by 1 point
\item Lowers charisma rating permanently by \ point
\item Lowers intelligence rating permanently by 1 point
\item Cures all damage on one character
\item Causes invisibility for 1--6 hours of game time (subject to
normal restrictions)
\item Poison (saving throw at +1)
\item Makes a 500 g.p. gem (pearl) appear in character's
hand
\item Gives a permanent +1 to any single weapon carried by
character (if more than one now carried, roll randomly to
determine which)
\item Heals all lost hit points of character (if any)
\item Causes idiocy for 1--4 hours (unable to function intelligently
or fight, must be led by other adventurers)
\item Gives a special one-time bonus of 1--6 hit points to the
character (these are the first ones lost the next time damage
or injury is taken)
\item Gives a curse: the character will sleep for 72 hours
straight each month, beginning one day before and
ending one day after each new moon (can only be removed
by a remove curse spell)
\item Has no effect
\end{enumerate}

Monster:

Treasure \& Location:

\subsection{SUNKEN CAVERN}
This small cavern lies at the bottom of
a short, sloping corridor. The walls are wet with moisture, and
glisten in any reflected light.

Monster:

Treasure \& Location:

\subsection{CAVERN}

Monster:

Treasure \& Location:

\subsection{ARENA CAVERN}
This cavern, designed as a small theatre
or arena, is unfinished. The center portion of the room is
sunken about 15' below the floor level, and the sides slope
downward from the surrounding walls to form a small amphitheatre.

Monster:

Treasure \& Location:

\subsection{PHOSPHORESCENT CAVE}
This medium-sized cavern
and its irregularly-shaped eastern arm present an eerie sight
to explorers. A soft phosphorescent glow bathes the entire
area independent of any other illumination, and the strange
light is caused by the widespread growth (on walls, ceiling,
and even parts of the floor) of a light purplish mold. The mold
itself is harmless.

Monster:

Treasure \& Location:

\subsection{WATER PIT}
This room contains the 8' deep pool of water
into which any unwary adventurers are precipitated from the
trap on the upper level (see the special description of the
trap under the description of room 36). As described there,
the water is extremely cold. Anyone entering the water
(whether voluntarily or not) must spend a full hour recovering
from its chilly effects.
The pool is about 20' across and is filled by a cold spring.

Monster:

Treasure \& Location:

\subsection{SIDE CAVERN}
This cavern is unusual only in that its eastern
rock wall is striated with irregular diagonal streaks of a
bluish ore (of no unusual use or value to the adventurers).

Monster:

Treasure \& Location:

\subsection{RAISED CAVERN}
This room, off the southeast corner of
the grand cavern, is accessible by climbing four upward
steps. Its eastern wall also shows diagonal streaks of the
same bluish ore noticeable in room 51. The room has a low
ceiling (only 5'), so some humans may find it difficult to stand
fully erect.

Monster:

Treasure \& Location:

\subsection{GRAND CAVERN OF THE BATS}
This majestic cave is the
largest in the complex, and is impressive due to its size and
volume, for the ceiling is almost 60' above. A corridor sloping
downward into the cavern (noticeable even by nondwarves)
gives primary access to the room on its south wall.
A secondary entrance/exit is via a secret door to the west,
while steps to the southeast lead up to room 52.

\end{document}
